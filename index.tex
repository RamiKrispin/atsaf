% Options for packages loaded elsewhere
\PassOptionsToPackage{unicode}{hyperref}
\PassOptionsToPackage{hyphens}{url}
\PassOptionsToPackage{dvipsnames,svgnames,x11names}{xcolor}
%
\documentclass[
  letterpaper,
  DIV=11,
  numbers=noendperiod]{scrreprt}

\usepackage{amsmath,amssymb}
\usepackage{lmodern}
\usepackage{iftex}
\ifPDFTeX
  \usepackage[T1]{fontenc}
  \usepackage[utf8]{inputenc}
  \usepackage{textcomp} % provide euro and other symbols
\else % if luatex or xetex
  \usepackage{unicode-math}
  \defaultfontfeatures{Scale=MatchLowercase}
  \defaultfontfeatures[\rmfamily]{Ligatures=TeX,Scale=1}
\fi
% Use upquote if available, for straight quotes in verbatim environments
\IfFileExists{upquote.sty}{\usepackage{upquote}}{}
\IfFileExists{microtype.sty}{% use microtype if available
  \usepackage[]{microtype}
  \UseMicrotypeSet[protrusion]{basicmath} % disable protrusion for tt fonts
}{}
\makeatletter
\@ifundefined{KOMAClassName}{% if non-KOMA class
  \IfFileExists{parskip.sty}{%
    \usepackage{parskip}
  }{% else
    \setlength{\parindent}{0pt}
    \setlength{\parskip}{6pt plus 2pt minus 1pt}}
}{% if KOMA class
  \KOMAoptions{parskip=half}}
\makeatother
\usepackage{xcolor}
\setlength{\emergencystretch}{3em} % prevent overfull lines
\setcounter{secnumdepth}{5}
% Make \paragraph and \subparagraph free-standing
\ifx\paragraph\undefined\else
  \let\oldparagraph\paragraph
  \renewcommand{\paragraph}[1]{\oldparagraph{#1}\mbox{}}
\fi
\ifx\subparagraph\undefined\else
  \let\oldsubparagraph\subparagraph
  \renewcommand{\subparagraph}[1]{\oldsubparagraph{#1}\mbox{}}
\fi


\providecommand{\tightlist}{%
  \setlength{\itemsep}{0pt}\setlength{\parskip}{0pt}}\usepackage{longtable,booktabs,array}
\usepackage{calc} % for calculating minipage widths
% Correct order of tables after \paragraph or \subparagraph
\usepackage{etoolbox}
\makeatletter
\patchcmd\longtable{\par}{\if@noskipsec\mbox{}\fi\par}{}{}
\makeatother
% Allow footnotes in longtable head/foot
\IfFileExists{footnotehyper.sty}{\usepackage{footnotehyper}}{\usepackage{footnote}}
\makesavenoteenv{longtable}
\usepackage{graphicx}
\makeatletter
\def\maxwidth{\ifdim\Gin@nat@width>\linewidth\linewidth\else\Gin@nat@width\fi}
\def\maxheight{\ifdim\Gin@nat@height>\textheight\textheight\else\Gin@nat@height\fi}
\makeatother
% Scale images if necessary, so that they will not overflow the page
% margins by default, and it is still possible to overwrite the defaults
% using explicit options in \includegraphics[width, height, ...]{}
\setkeys{Gin}{width=\maxwidth,height=\maxheight,keepaspectratio}
% Set default figure placement to htbp
\makeatletter
\def\fps@figure{htbp}
\makeatother
\newlength{\cslhangindent}
\setlength{\cslhangindent}{1.5em}
\newlength{\csllabelwidth}
\setlength{\csllabelwidth}{3em}
\newlength{\cslentryspacingunit} % times entry-spacing
\setlength{\cslentryspacingunit}{\parskip}
\newenvironment{CSLReferences}[2] % #1 hanging-ident, #2 entry spacing
 {% don't indent paragraphs
  \setlength{\parindent}{0pt}
  % turn on hanging indent if param 1 is 1
  \ifodd #1
  \let\oldpar\par
  \def\par{\hangindent=\cslhangindent\oldpar}
  \fi
  % set entry spacing
  \setlength{\parskip}{#2\cslentryspacingunit}
 }%
 {}
\usepackage{calc}
\newcommand{\CSLBlock}[1]{#1\hfill\break}
\newcommand{\CSLLeftMargin}[1]{\parbox[t]{\csllabelwidth}{#1}}
\newcommand{\CSLRightInline}[1]{\parbox[t]{\linewidth - \csllabelwidth}{#1}\break}
\newcommand{\CSLIndent}[1]{\hspace{\cslhangindent}#1}

\KOMAoption{captions}{tableheading}
\makeatletter
\makeatother
\makeatletter
\@ifpackageloaded{bookmark}{}{\usepackage{bookmark}}
\makeatother
\makeatletter
\@ifpackageloaded{caption}{}{\usepackage{caption}}
\AtBeginDocument{%
\ifdefined\contentsname
  \renewcommand*\contentsname{Table of contents}
\else
  \newcommand\contentsname{Table of contents}
\fi
\ifdefined\listfigurename
  \renewcommand*\listfigurename{List of Figures}
\else
  \newcommand\listfigurename{List of Figures}
\fi
\ifdefined\listtablename
  \renewcommand*\listtablename{List of Tables}
\else
  \newcommand\listtablename{List of Tables}
\fi
\ifdefined\figurename
  \renewcommand*\figurename{Figure}
\else
  \newcommand\figurename{Figure}
\fi
\ifdefined\tablename
  \renewcommand*\tablename{Table}
\else
  \newcommand\tablename{Table}
\fi
}
\@ifpackageloaded{float}{}{\usepackage{float}}
\floatstyle{ruled}
\@ifundefined{c@chapter}{\newfloat{codelisting}{h}{lop}}{\newfloat{codelisting}{h}{lop}[chapter]}
\floatname{codelisting}{Listing}
\newcommand*\listoflistings{\listof{codelisting}{List of Listings}}
\makeatother
\makeatletter
\@ifpackageloaded{caption}{}{\usepackage{caption}}
\@ifpackageloaded{subcaption}{}{\usepackage{subcaption}}
\makeatother
\makeatletter
\@ifpackageloaded{tcolorbox}{}{\usepackage[many]{tcolorbox}}
\makeatother
\makeatletter
\@ifundefined{shadecolor}{\definecolor{shadecolor}{rgb}{.97, .97, .97}}
\makeatother
\makeatletter
\makeatother
\ifLuaTeX
  \usepackage{selnolig}  % disable illegal ligatures
\fi
\IfFileExists{bookmark.sty}{\usepackage{bookmark}}{\usepackage{hyperref}}
\IfFileExists{xurl.sty}{\usepackage{xurl}}{} % add URL line breaks if available
\urlstyle{same} % disable monospaced font for URLs
\hypersetup{
  pdftitle={Applied Time Series Analysis and Forecasting with R},
  pdfauthor={Rami Krispin},
  colorlinks=true,
  linkcolor={blue},
  filecolor={Maroon},
  citecolor={Blue},
  urlcolor={Blue},
  pdfcreator={LaTeX via pandoc}}

\title{Applied Time Series Analysis and Forecasting with R}
\author{Rami Krispin}
\date{October 01, 2022}

\begin{document}
\maketitle
\ifdefined\Shaded\renewenvironment{Shaded}{\begin{tcolorbox}[breakable, interior hidden, enhanced, borderline west={3pt}{0pt}{shadecolor}, boxrule=0pt, sharp corners, frame hidden]}{\end{tcolorbox}}\fi

\renewcommand*\contentsname{Table of contents}
{
\hypersetup{linkcolor=}
\setcounter{tocdepth}{2}
\tableofcontents
}
\bookmarksetup{startatroot}

\hypertarget{preface}{%
\chapter*{Preface}\label{preface}}
\addcontentsline{toc}{chapter}{Preface}

Welcome to the website of \textbf{Applied Time Series Analysis and
forecasting with R}! This is an early ``work in progress'', and the book
chapters will be added gradually in the coming months. Thank you for
your patience.

As its name implies, this book focuses on applied methods for handling
and analyzing time series data and building forecasting models using R.
That includes working with time series data and objects, using data
visualization methods to explore the data, and using statistical methods
to generate a forecast. In addition, we will spend some time on
approaches for scaling and productionize your work by using fun
examples.

\hypertarget{audience}{%
\section*{Audience}\label{audience}}
\addcontentsline{toc}{section}{Audience}

This book assumes that you don't have any previous background in time
series analysis and forecasting but have some basic knowledge in
statistics, probability, regression analysis, and R programming. While I
will cover some of the basic theories beyond the methods and approaches
of time series, the focus of this book is more applied applications of
time series and forecasting.

\hypertarget{roadmap}{%
\section*{Roadmap}\label{roadmap}}
\addcontentsline{toc}{section}{Roadmap}

The book's first version will cover the foundation of time series
analysis, focusing on time series data and objects and descriptive
methods for analyzing time series data. The following versions will
include additional layers covering different forecasting approaches and
other topics. Below are the book's core milestones:

\begin{itemize}
\tightlist
\item
  \texttt{V1} - Foundation of time series analysis
\item
  \texttt{V2} - Traditional time series forecasting methods (Smoothing
  methods, ARIMA, Linear Regression)
\item
  \texttt{V3} - Advanced regression methods (GLM, GAM, etc.)
\item
  \texttt{V4} - Bayesian forecasting approaches
\item
  \texttt{V5} - Machine and deep learning methods
\item
  \texttt{V6} - Scaling and production approaches
\end{itemize}

More details are available on the book
\href{https://github.com/RamiKrispin/atsaf\#table-of-contents}{Github
page}, and you can track the progress on the book's
\href{https://github.com/users/RamiKrispin/projects/4}{project page}.

\hypertarget{reproducability}{%
\section*{Reproducability}\label{reproducability}}
\addcontentsline{toc}{section}{Reproducability}

The book development is done inside a dockerized environment to ensure a
high level of reproducibility. The book's development environment can be
found on
\href{https://hub.docker.com/repository/docker/rkrispin/atsaf}{Docker
Hub}, and you can pull and run the image locally by using:

\begin{verbatim}
docker pull rkrispin/atsaf:dev.0.0.0.9000
\end{verbatim}

\hypertarget{resources}{%
\section*{Resources}\label{resources}}
\addcontentsline{toc}{section}{Resources}

All the book source code can be found
\href{https://github.com/RamiKrispin/atsaf}{here}, and get updates on
the book's progress on Twitter and Telegram:

The book website was created with \href{https://quarto.org/}{Quarto}.

\hypertarget{license}{%
\section*{License}\label{license}}
\addcontentsline{toc}{section}{License}

This book is licensed under a
\href{https://creativecommons.org/licenses/by-nc-sa/4.0/}{Creative
Commons Attribution-NonCommercial-ShareAlike 4.0 International} License.

\bookmarksetup{startatroot}

\hypertarget{introduction}{%
\chapter{Introduction}\label{introduction}}

Introduction chapter - placeholder.

\bookmarksetup{startatroot}

\hypertarget{prerequisites}{%
\chapter{Prerequisites}\label{prerequisites}}

Prerequisites chapter - placeholder.

\bookmarksetup{startatroot}

\hypertarget{dates-and-time-objects}{%
\chapter{Dates and Time Objects}\label{dates-and-time-objects}}

Dates and Time Objects chapter - placeholder.

\bookmarksetup{startatroot}

\hypertarget{the-ts-class}{%
\chapter{The ts Class}\label{the-ts-class}}

The ts Class chapter - placeholder.

\bookmarksetup{startatroot}

\hypertarget{the-timetk-class}{%
\chapter{The timetk Class}\label{the-timetk-class}}

The timetk Class chapter - placeholder.

\bookmarksetup{startatroot}

\hypertarget{the-tsibble-class}{%
\chapter{The tsibble Class}\label{the-tsibble-class}}

The tsibble Class chapter - placeholder.

\bookmarksetup{startatroot}

\hypertarget{plotting-time-series-objects}{%
\chapter{Plotting Time Series
Objects}\label{plotting-time-series-objects}}

Plotting Time Series Objects chapter - placeholder.

\bookmarksetup{startatroot}

\hypertarget{seasonal-analysis}{%
\chapter{Seasonal Analysis}\label{seasonal-analysis}}

Seasonal Analysis chapter - placeholder.

\bookmarksetup{startatroot}

\hypertarget{correlation-analysis}{%
\chapter{Correlation Analysis}\label{correlation-analysis}}

Correlation Analysis chapter - placeholder.

\bookmarksetup{startatroot}

\hypertarget{smoothing-methods}{%
\chapter{Smoothing Methods}\label{smoothing-methods}}

Smoothing Methods chapter - placeholder.

\bookmarksetup{startatroot}

\hypertarget{time-series-decomposition}{%
\chapter{Time Series Decomposition}\label{time-series-decomposition}}

Time Series Decomposition chapter - placeholder.

\bookmarksetup{startatroot}

\hypertarget{summary}{%
\chapter{Summary}\label{summary}}

In summary, this book has no content whatsoever.

\bookmarksetup{startatroot}

\hypertarget{references}{%
\chapter*{References}\label{references}}
\addcontentsline{toc}{chapter}{References}

\hypertarget{refs}{}
\begin{CSLReferences}{0}{0}
\end{CSLReferences}



\end{document}
